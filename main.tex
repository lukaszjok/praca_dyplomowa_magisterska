%%%%%%%%%%%%%%%%%%%%%%%%%%%%%%%%%%%%%%%%%
% Simple Sectioned Essay Template
% LaTeX Template
%
% This template has been downloaded from:
% http://www.latextemplates.com
%
% Note:
% The \lipsum[#] commands throughout this template generate dummy text
% to fill the template out. These commands should all be removed when 
% writing essay content.
%
%%%%%%%%%%%%%%%%%%%%%%%%%%%%%%%%%%%%%%%%%

%----------------------------------------------------------------------------------------
%	PACKAGES AND OTHER DOCUMENT CONFIGURATIONS
%----------------------------------------------------------------------------------------

\documentclass[12pt]{article} % Default font size is 12pt, it can be changed here


\usepackage{polski}
\usepackage[utf8]{inputenc}

\usepackage{geometry} % Required to change the page size to A4
\geometry{a4paper} % Set the page size to be A4 as opposed to the default US Letter
\newgeometry{tmargin=2cm, bmargin=2cm, lmargin=2cm, rmargin=2cm}

\usepackage{graphicx} % Required for including pictures

\usepackage{float} % Allows putting an [H] in \begin{figure} to specify the exact location of the figure
\usepackage{wrapfig} % Allows in-line images such as the example fish picture

\usepackage{lipsum} % Used for inserting dummy 'Lorem ipsum' text into the template

\linespread{1.2} % Line spacing

%\setlength\parindent{0pt} % Uncomment to remove all indentation from paragraphs

\graphicspath{{Pictures/}} % Specifies the directory where pictures are stored
\usepackage{graphicx}

\begin{document}

%----------------------------------------------------------------------------------------
%	TITLE PAGE
%----------------------------------------------------------------------------------------

\begin{titlepage}

\newcommand{\HRule}{\rule{\linewidth}{0.5mm}} % Defines a new command for the horizontal lines, change thickness here

\center % Center everything on the page

\textsc{\LARGE Politechnika Wrocławska}\\[1.5cm] % Name of your university/college
\textsc{\Large Praca dyplomowa}\\[0.5cm] % Major heading such as course name
\textsc{\large Optimizing routes and commuting time of an ambulance to the victim.}\\[0.5cm] % Minor heading such as course title

\HRule \\[0.4cm]
{ \huge \bfseries 
Optymalizacja drogi i czasu dojazdu karetki pogotowia do poszkodowanego. }\\[0.4cm] % Title of your document
\HRule \\[1.5cm]

\begin{minipage}{0.4\textwidth}
\begin{flushleft} \large
\emph{Autor:}\\
Łukasz \textsc{Joksch}(200963) \\

\end{flushleft}
\end{minipage}
~
\begin{minipage}{0.4\textwidth}
\begin{flushright} \large
\emph{Opiekun:} \\
dr Wojciech \textsc{Kmiecik} % Supervisor's Name
\end{flushright}
\end{minipage}\\[4cm]

{\large \today}\\[3cm] % Date, change the \today to a set date if you want to be precise

%\includegraphics{Logo}\\[1cm] % Include a department/university logo - this will require the graphicx package

\vfill % Fill the rest of the page with whitespace

\end{titlepage}

%----------------------------------------------------------------------------------------
%	TABLE OF CONTENTS
%----------------------------------------------------------------------------------------

\tableofcontents % Include a table of contents

\newpage % Begins the essay on a new page instead of on the same page as the table of contents 

%----------------------------------------------------------------------------------------
%	INTRODUCTION
%----------------------------------------------------------------------------------------

\section{Wstęp}
W swojej pracy chciałbym zaproponować nowy, własny model zarządzania zgłoszeniami ratunkowymi. Aktualny model posiada wiele wad, najpoważniejszą jest fakt, uzależnienia wyboru karetki od jej przynależenia do danej jednostki administracyjnej, nie zawsze jest to optymalny wybór. Przez takie podejście czas oczekiwania na pomoc przez pacjenta wydłuża się. Ponadto, wielokrotnie kierowca, sam decyduje o trasie przejazdu, może wspierać się systemem GPS i nawigacją ale nie ma on pewności, ani wpływu na to czy wybór ten jest optymalny. Decyzja o drodze zostaje podjęta często w chwili, gdy kierowca już prowadzi pojazd, a nie jak intuicyjnie mogłoby wydawać się przed wejściem do karetki przez dedykowany system.
Celem mojej pracy byłoby zaproponowanie nowego modelu wyboru karetki wysyłanej do zgłoszenia oraz implementacja aplikacji wspomagającej tą czynność. W celu przeprowadzenia badań zaimplementowane zostaną różne algorytmy wyboru trasy, uzależniając jej wybór od stopnia zakorkowania danej trasy. Uzyskane wyniki zostaną ze sobą porównane, by w rezultacie wyłonić ten optymalny.
\section{Cel pracy}
W swojej pracy magisterskiej skupię się przedstawieniem problemu dotarcia karetki pogotowia do pacjenta wzywającego pomocy.
Na początku pracy zostanie przedstawiona aktualna sytuacja w naszym kraju dotycząca systemu ratownictwa oraz pokazane zostanie obecne podejście zagospodarowania i organizacji zespołów ratowniczych. W celu uwiarygodnienia, ale przede wszystkim unaocznieniu poruszanego problemu przytoczone zostaną raporty ministerstwa zdrowia i Najwyższej Izby Kontroli (NIK). Zawarte w nich dane pokażą skalę problemu oraz zobrazują obecny model. W swojej pracy postaram pokazać się, iż ten model nie jest optymalny, tj. nie
gwarantuje najszybszego dotarcia do pacjenta w możliwie krótkim czasie. Ponadto system w rozumieniu globalnym działa nierównomiernie, tzn. przy aktualnym gospodarowaniu karetkami w obszarze województwa może dojść do sytuacji, w której w przypadku gdy istnieje wzmożone zapotrzebowanie karetek w jednym województwie, karetki z innego województwa muszą być organizowane na drodze próśb o użyczenie, a nie są brane pod uwagę w momencie zaistnienia zgłoszenia. Dodatkowo należy wspomnieć, iż wykorzystywane mierniki skuteczności zespołów ratowniczych (czas dojazdu do poszkodowanego) niosą za sobą mało informacji.
W swojej pracy będę chciał przedstawić nowy model organizacji karetek pogotowia, czyli taki nie wynikający z stricte z podziałów administracyjnych, a optymalny pod względem czasu i drogi dojazdu z punktu startowego do docelowego.
W kolejnej części swojej pracy dyplomowej zawrę informację o algorytmach optymalizujących drogę. Przedstawione zostaną matematyczne modele oraz pseudokod pokazujący ich programistyczne rozwinięcie. Szczególna uwaga zostanie skupiona na czasie działania każdego algorytmu oraz stopniu trudności jego implementacji i złożoności obliczeniowej.

\section{Zakres pracy}
Rezultatem mojej pracy będzie program komputerowy, w którym owe algorytmy zostaną zaimplementowane. W celach badawczych możliwym będzie wybór, z którego algorytmu program ma w danej chwili korzystać, tj. wyznaczać drogę ze Szpitala do pacjenta. Przewiduje się, że program będzie pobierał dane o korkach drogowych z Internetu, co pozwoli uniknąć zatłoczonych dróg. Będzie to nieocenione dla kierowców. Należy zaznaczyć, iż proponowany program nie będzie jedynie kolejną aplikacją typu GPS, bowiem będzie posiadał zaimplementowaną listę punktów startowych (szpitali, miejsc organizacji karetek pogotowia) i w zależności od miejsca wezwania, sam wybierze z której stacji najlepiej wysłać pogotowie.
Oprogramowanie wspierające symulację, obliczenia i prezentujące wyniki badań zostanie zaprogramowane w języku JAVA, przewiduję powstanie wersji graficznej programu w celu przejrzystej prezentacji badań oraz łatwego poruszania się po jego interfejsie.
Zaimplementowane algorytmy będą porównywane przede wszystkim pod względem czasu pracy, przewidywanego czasu dojazdu karetki do pacjenta oraz długości trasy. Wyniki badań zostaną zapisane do plików w postaci danych tabelarycznych i wykresów. W celu łatwego porównania danej trasy wyniki badań dla każdego algorytmu przedstawione zostaną na jednym wykresie.
Po procesie generowania danych i symulacji przeprowadzona zostanie dogłębna analiza mająca na celu wskazanie optymalnego rozwiązania lub wskazania tego rozwiązania, które na tle pozostałych jest wyjątkowo nieefektywne.
\section{Tematyka pracy dyplomowej - ETAP 3.}

\subsection{Podstawowe Pojęcia}
Tak jak zostało to zasygnalizowane we wstępie głównym celem tej pracy dyplomowej jest wypracowanie nowego modelu zarządzania karetkami pogotowia. Chcąc uczynić proponowane rozwiązanie możliwie najbardziej użytecznym, wygodnym i funkcjonalnym należy dokładnie przyjrzeć się aktualnie działającemu systemowi powiadamiania ratunkowego i jego architekturze. Jest to koniecznym ze względu na to, iż tylko w ten sposób można wychwycić jego wady, niedoskonałości, które mogą zostać poprawione.\\
Chcąc dokładnie przedstawić aktualną architekturę, słowem wstępu, poniżej przedstawiono najważniejsze elementy systemu powiadamiania ratunkowego:
\begin{itemize}
\item 
System 112 - 
Poprzez pojęcie system 112 - jednolity krajowy system odbioru zgłoszeń nanumer alarmowy 112. System ten obejmuje sieć Centrów Powiadamiania Ratunkowego (CPR) - opisane poniżej.Z systemem
współpracują: Policja, Państwowa Straż Pożarna oraz Pogotowie Ratunkowe, a także inne jednostki ratownicze, do których przekierowywane są zgłoszenia alarmowe, które potwierdzają podjęcie akcji ratowniczej lub innej interwencji. Dodatkowo z systemem współpracują publiczni operatorzy sieci telefonicznych oraz
dostawcy ogólnodostępnych usług telekomunikacyjnych.
\item 
Centrum powiadamiania ratunkowego (skrót CPR, ang. public-safety answering point, PSAP) – jednolity system obsługujący zgłoszenia alarmowe kierowane do numerów alarmowych 112, 997, 998 i 999,który  umożliwia przekazanie zgłoszenia alarmowego do odpowiednich służb ratowniczych. 

\item 
Operator 112 - osoba pracująca w CPR, której zadaniem jest odbieranie zgłoszeń od dzwoniących na numer 112, a w następstwie przekazująca je do właściwych służb.
\item 
Dyspozytor - osoba, która dysponuje siłami i środkami pozwalającymi nieść pomoc poszkodowanym w strukturach Policji, Państwowej Straży Pożarnej, Pogotowia Ratunkowego.


\end{itemize}
Centrum powiadamiania ratunkowego (skrót CPR, ang. public-safety answering point, PSAP) – jednolity system obsługujący zgłoszenia alarmowe kierowane do numerów alarmowych 112, 997, 998 i 999,który  umożliwia przekazanie zgłoszenia alarmowego do odpowiednich służb ratowniczych. 

\subsection{Obecny model i architektura Centrum Powiadamiania Ratunkowego}
Aktualne podejście architektoniczne systemu jest podejściem scentralizowanym (choć posiada kilkanaście centrali - jednak połączonych między sobą i ściśle współpracujących), związane z podziałem administracyjnym Polski na województwa. Poszczególne Centra  Powiadamiania Ratunkowego są połączone ze sobą w sieć przy użyciu światłowodów, co pozwala na szybkie przesyłanie danych między tymi ośrodkami. Ponadto dysponują one
% [przypis_duzo_info]
% [potwierdzić informacje informatyczne]
wyspecjalizowanymi serwerami, co jest szczególnie istotną informacją w kontekście tej pracy dyplomowej, gdyż pozwoli to na szybsze działanie proponowanej aplikacji. Platforma aplikacyjna wykorzystywana obecnie pozwala na przesyłanie danych między operatorami 112 a dyspozytorami, dzięki czemu rezultaty pracy aplikacji optymalizującej drogę i czas karetki pogotowia, będą mogły być przekazywane między poszczególnymi jednostkami w postaci plików.
W ramach systemu 112 mamy 16 wojewódzkich Centrów Powiadamiania Ratunkowego. System kooperuje z Platformą Lokalizacyjno-Informacyjną, którą zarządza Urząd Komunikacji Elektronicznej. Poniżej pokazano rozmieszczenie oddziałów CPR na terenie Polski.

\begin{figure}[H]
  \centering
  \includegraphics[width=\columnwidth]{images/mapa_cpr.png}
  \caption{Rozmieszczenie Centrów Powiadamiania Ratunkowego}
\end{figure}

\subsection{Funkcjonowanie Systemu Powiadamiania Ratunkowego}

Osoba dzwoniąca pod ratunkowy numer 112, zgłaszając potrzebę wezwania pogotowia ratunkowego łączy się z dyspozytorem CPR. Ten posiada dedykowany system informatyczny na swoim stanowisku - w jego skład wchodzą specjalistyczne oprogramowanie zarządzające zebranymi danymi od osoby zgłaszającej wezwanie, dane o abonencie urządzenia z którego dzwoniący kontaktuje się z CPR oraz aplikację będącą platformą lokalizująco-infrmacjną. Dzięki takiemu wyposażeniu dyspozytor jest w stanie bardzo dokładnie zlokalizować osobę dzwoniącą - na mocy odpowiednich ustaw czas lokalizacji abonenta nie powinien przekroczyć 4 sekund. 
% [sprawdzic jak to jest dokladnie]
Zebrawszy wszystkie niezbędne informacje operator 112 przekazuje zdarzenie do odpowiedniego dyspozytora wybieranego w sposób arbitralny. Aktualnie nie funkcjonuje żaden system wspierający wybór optymalnej stacji pogotowia pod względem czasu i drogi dojazdu do poszkodowanego. Co więcej, dyspozytor wybiera w pierwszej kolejności stację pogotowia z danego województwa. Takie podejście może powodować, że czas dojazdu do wzywającego pomoc może znacząc się wydłużyć. W związku z taką sytuacją zaproponowane zostanie nowe podejście, które opisano poniżej.

\subsection{Propozycja nowego rozwiązania}
Chcąc zoptymalizować drogę i czas dojazdu karetki pogotowia należy zaimplementować rozwiązanie aplikacyjne działające już na etapie rejestracji zgłoszenia u operatora 112. W takiej sytuacji, w chwili przeprowadzania wstępnego wywiadu dotyczącego zaistniałego zdarzenia, oprogramowanie mogłoby wyznaczać optymalną trasę dojazdu do wzywającego pogotowie. Takie dane byłyby przekazane drogą elektroniczną do dyspozytora, który mógłby ewentualnie skorygować trasę, a w następstwie przekazać ją kierowcy karetki. Dane o dojeździe przekazywane byłyby w formie mapy graficznej, wiadomości tekstowej zawierającej wykaz kolejnych miejscowości i dróg lub głosowe wytyczne dyspozytora. 

\subsection{Algorytm nowego podejścia}
W rozwiązaniu proponowanym w pracy dyplomowej przewiduje się, że w chwili uzyskania informacji o położeniu poszkodowanego - operator 112 lub dyspozytor umieszcza takie dane w programie jako parametr wejściowy. Następnie aplikacja poszukuje najbliższej stacji pogotowia   badając promień odległości od miejsca zgłoszenia, zwiększając go o zadaną wartość(krok promienia poszukiwań), unikając tym samym barier administracyjnych. W kolejnym etapie, mając ustalony punkt startowy i końcowy - aplikacja wyznaczy optymalna tracę przejazdu karetki. W pracy dyplomowej badany będzie problem komiwojażera i różne algorytmy próbujące rozwiązać zagadnienie, szczegóły przedstawione zostaną w kolejnych punktach. Dodatkowo aplikacja będzie zbierała informacje o korkach drogowych - ta informacja będzie parametrem uwzględnianym w implementacji co pozwoli  na szybkie dotarcie do poszkodowanego.

\subsection{Problem Komiwojażera}
%[kopiuj/wkliej :)] - do poprawy - wikipedia
Problem komiwojażera (ang. travelling salesman problem, TSP) – zagadnienie optymalizacyjne, polegające na znalezieniu minimalnego cyklu Hamiltona w pełnym grafie ważonym.
Nazwa pochodzi od typowej ilustracji problemu, przedstawiającej go z punktu widzenia wędrownego sprzedawcy (komiwojażera): dane jest n miast, które komiwojażer ma odwiedzić, oraz odległość / cena podróży / czas podróży pomiędzy każdą parą miast. Celem jest znalezienie najkrótszej / najtańszej / najszybszej drogi łączącej wszystkie miasta, zaczynającej się i kończącej się w określonym punkcie.
Symetryczny problem komiwojażera (STSP) polega na tym, że dla dowolnych miast A i B odległość z A do B jest taka sama jak z B do A. W asymetrycznym problemie komiwojażera (ATSP) odległości te mogą być różne.
Rozwinięciem problemu komiwojażera jest problem marszrutyzacji.\\

%[kopiuj/wkliej :)] - do poprawy - wikipedia
Problem marszrutyzacji - problem decyzyjny polegający na wyznaczeniu optymalnych tras przewozowych dla pewnej ściśle określonej liczby środków transportu, której zadaniem jest obsłużenie zbioru klientów znajdujących się w różnych punktach przy zachowaniu ograniczeń. Kryterium optymalizacji jest całkowity koszt transportu (wyrażony odległościowo, cenowo lub czasowo). Istnieją również rozwinięcia problemu uwzględniające więcej, niż jedno kryterium optymalizacji. Problem marszrutyzacji należy do podstawowej problematyki zarządzania operacyjnego flotą środków transportu (rzadziej zarządzania na wyższym szczeblu).




\subsubsection{Algorytm mrówkowy}
%[kopiuj/wkliej :)] - do poprawy
Mrówki są owadami żyjącymi w koloniach, ich przezycie jest uzależnione od wspólnego działania. Ich misja jest budowa mrowiska oraz
poszukiwanie pożywienia. Pojedynczy osobnik nie jest w stanie tego dokonać, dopiero w grupie mrówki mogą wykazać się swoja inteligencja.
Aby przeżyć, niezbędna jest współpraca miedzy wszystkimi jednostkami
kolonii. Mrówki wykazują złożone zachowania społeczne, które od dawna przyciągają uwagę człowieka. W dzieciństwie z pewnoscia wiele osób usiłowało utrudnić mrówkom „prace” poprzez umieszczenie jakiejś przeszkody na ich drodze, aby sprawdzić ich reakcje na takie zaburzenie. Można w ten sposób zauważyć, ze owady te bez problemu radzą sobie w takich sytuacjach . Początkowo ich ruch jest chaotyczny, jednak z czasem wypracowują one najkrótsza droge. Jak to sie dzieje? Mianowicie odbywa się to za sprawa komunikacji za pomocą feromonów, czyli związków infochemicznych pozostawianych w środowisku. Mrówki wędrując w poszukiwaniu pożywienia wybierają losowe trasy.
Kiedy jedna z nich znajdzie pokarm, wracając z nim do mrowiska
na całej ścieżce wydziela feromon. Kolejne mrówki, dzięki zdolności wyczuwania tych feromonów podążają droga, która przeszło najwięcej jej
towarzyszy, również zostawiając ślad po sobie. Gdy na drodze pojawi sie przeszkoda, mrówki muszą wybrać pomiędzy skrętem w prawo lub
w lewo, przy czym prawdopodobieństwo wyboru jednej z tych opcji jest
jednakowe. Osobniki, które przypadkowo wybierają krótsza ścieżkę,
utrwalaja na niej slad feromonowy (mniej feromonu zdarzyło odparować,
podczas gdy szybciej wyparowuje on na ścieżce dłuższej). Innymi słowy,
feromon szybciej osadza sie na krótszej drodze. W rezultacie,
wszystkie mrówki wybieraja krótsza trasę, ponieważ podążając szlakiem
zapachowym w kierunku pożywienia, dokładają również swój feromon
do już istniejącego. Poziom zapachu na tej drodze staje sie tak wysoki,
ze podażą nią cała kolonia. Feromony z czasem wyparowują, tracąc
swa intensywnosc, dlatego gdy źródło pożywienia sie wyczerpie, ścieżka
znika.

\subsubsection{Algorytm genetyczny}
Inspiracja do powstania algorytmu genetycznego była natura i zachodzące
w niej zjawiska. Algorytm ten do poszukiwania optymalnych
rozwiązań naśladuje naturalne procesy związane z ewolucja, takie jak
dziedziczenie genetyczne. Wszystkie organizmy żywe żyją w stale zmieniającym
się świecie, cześć z nich jest lepiej przystosowana do danego
środowiska, cześć gorzej. Organizmy radzące sobie lepiej maja większą
szanse na przeżycie. Przykładem moze tu byc kot polujacy na myszy.
Szybki, zwinny i sprytny kot szybciej upoluje mysz, anizeli wolny, niezdarny.
Stad ten pierwszy ma większe szanse na przeżycie, a także przekazanie
swoich „lepszych” genów kolejnym pokoleniom. Oczywiście czesc
wolnych kotów tez przezyje, wprowadzając mieszaninę materiału genetycznego.
Jednak w naturalny sposób, cała populacja w procesie ewolucji
dąży do ulepszenia, organizmy „lepsze” reprodukują się, a „gorsze” wymierają.
Opisana zależność wykorzystywana jest w działaniu algorytmów
genetycznych
\section{Opracowanie pracy magisterskiej}

Przygotowując się do redakcji pracy magisterskiej ustalone zostały niniejsze fakty, które spisano w kolejnych podpunktach.

\subsection{Aspekt Badawczy}
Analiza właściwości i charakterystyk zaproponowanych algorytmów służących do rozwiązania rozważanego problemu optymalizacyjnego na podstawie wyników otrzymanych przy wykorzystaniu zaimplementowanego systemu symulacyjnego. Rekomendacje związane z zastosowaniem zaproponowanych rozwiązań w rzeczywistych warunkach oraz propozycje dalszych kierunków badań.

\subsection{Aspekt Inżynierski}
Projekt i implementacja zaproponowanych algorytmów związanych z problemem wyznaczania optymalnej drogi przejazdu pojazdu karetki pogotowia do poszkodowanego. Implementacja systemu służącego do przeprowadzenia niezbędnych badań symulacyjnych.

\subsection{Spis treści pracy dyplomowej}

\begin{enumerate}
\item Wstęp\\
W tym rozdziale omówione zostaną cel, motywacja i ogólny zarys problemu. Nakreślony zostanie zakres pracy dyplomowej.

\item Podstawowe pojęcia, nazewnictwo.\\
W tym miejscu przybliżona i omówiona będzie semantyka słów, zwrotów i pojęć wykorzystywanych w obszarze optymalizacji i samego pogotowia ratunkowego.

\item Bieżąca sytuacja.\\
Tutaj zawarte zostaną informacje pokazującą sytuację bieżącą poruszanego zagadnienia w przestrzeni życia publicznego. Zaprezentowane zostaną zalety i uwypuklone zostaną wady, by w rezultacie pokazać sposób ich eliminacji.

\item Matematyczne sformułowanie problemu.\\
Jak sam tytuł wskazuje, rozdział ten poruszy zagadnienia matematyczne. Wskazany zostanie model matematyczny - opisywany przed odpowiednie wzory i algorytmy.

\item Przegląd literaturowy dostępnych rozwiązań dla omawianego problemu optymalizacyjnego.\\
W tym miejscu wskazany zostanie przegląd literaturowy pozycji, który zawarty będzie w bibliografii. 

\item Projekt oraz implementacja zaproponowanych algorytmów rozwiązujących problem optymalizacyjny.\\
Ten rozdział będzie zawierał opis podejścia programistycznego, zastosowane technologie i metodykę programowania. Za pomocą schematu blokowego UML zostanie pokazana architektura oprogramowania i zachodzące w niej przepływy danych/informacji i wzajemne zależności pomiędzy odpowiednimi blokami zadaniowymi.

\item Implementacja systemu symulacyjnego.\\
W tej części pracy dyplomowej spisane zostanie sprawozdanie z wykonanego środowiska testowego wraz z graficznym wskazaniem dostępych funkcji.

\item Sformułowanie scenariuszy testowych dla zaproponowanych algorytmów.\\
Podpunkt ten odnosił będzie się do scenariuszy testowych środowiska symulacyjnego i testowego. Zostaną sporządzone scenariusze badań, możliwe poprzez udostępnienie możliwości zmiany parametrów środowiska badawczego. Według tych scenariuszy przeprowadzane będą późniejsze badania.

\item Przeprowadzenie badań i analiza otrzymanych rezultatów.\\
Dział ten będzie jedną z ważniejszych części pracy dyplomowej. Otrzymane wyniki ze środowiska badawczego zostaną przestawione w postaci tabelarycznej, wstawione zostaną wykresy oraz ich opisy.

\item Wnioski oraz propozycje dalszych kierunków badań.
W tej części wysnute zostaną wnioski wysnute na podstawie przeprowadzanych doświadczeń, badań. To także miejsce na podsumowanie i końcowe refleksje.
\end{enumerate}

\newpage
\section{Wykres Gantta}
Poniżej został przedstawiony wykres Gantta, który będzie wyznacznikiem, punktem odniesienia w pracy nad pracą dyplomową.

\begin{figure}[H]
  \centering
  \includegraphics[scale=0.5]{images/g1.png}
  \caption{Wykres Gantta część 1.}
\end{figure}

\begin{figure}[H]
  \centering
  \includegraphics[scale=0.5]{images/g2.png}
  \caption{Wykres Gantta część 2.}
\end{figure}

\begin{figure}[H]
  \centering
  \includegraphics[scale=0.5]{images/g3.png}
  \caption{Wykres Gantta część 3.}
\end{figure}




\newpage

%----------------------------------------------------------------------------------------
%	BIBLIOGRAPHY
%----------------------------------------------------------------------------------------
\section{Bibliografia}

\begin{thebibliography}{99}
\bibitem{pa} 
Applegate, David L., “The traveling salesman problem : a computational study / David L. Applegate [et al.]”, Princeton ; Oxford : Princeton University Press, cop. 2006.
\bibitem{pa} 
Agnieszka JAKUBOWSKA, Katarzyna PIECHOCKA, „- WYBRANE ALGORYTMY W ZASTOSOWANIU DO PROBLEMU KOMIWOJAŻERA”, JOURNAL OF TRANSLOGISTICS 2015
\bibitem{pa} 
Ivan Brezina Jr., Zuzana Čičková, „Solving the Travelling Salesman Problem Using the Ant Colony Optimization”, Management Information Systems, Vol. 6 (2011), No. 4 pp. 010-014 Received 12 July 2010, Accepted 23 September 2011
\bibitem{pa} 
Neill E. Bowler, Thomas M. A. Fink, Robin C. Ball, “Characterization of the probabilistic traveling salesman problem”, June 13, 2003
\bibitem{pa} 
T.H. Cormen, Ch.E. Leiserson, R.L.Rivest, „Wprowadzenie do algorytmów”, Wyd. Naukowo-Techniczne Warszawa, 2001
\bibitem{pa} 
CHRIS WALSHAW, “A MULTILEVEL APPROACH TO THE TRAVELLING SALESMAN PROBLEM”, London, SE10 9LS, United Kingdom, (Received August 2000; revisions received May 2001, July 2001; accepted August 2001)
\bibitem{pa} 
Christian Blum, “Ant colony optimization: Introduction and recent trends”, ALBCOM, LSI, Universitat Politècnica de Catalunya, Jordi Girona 1-3, Campus Nord, 08034 Barcelona, Spain Accepted 11 October 2005
\bibitem{pa} 
Dweepna Garg and 2Saurabh Shah, „ANT COLONY OPTIMIZATION FOR SOLVING TRAVELING SALESMAN PROBLEM”, International Journal of Computer Science and System Analysis, Vol. 5, No. 1, January-June 2011
\bibitem{pa} 
Koncepcja Systemu 112, MSWiA, Warszawa, wrzesień 2007
\bibitem{pa} 
Encyklopedia Internetowa - Wikipedia, hasło: "Państwowe Ratownictwo Medyczne",
https://pl.wikipedia.org/wiki/Państwowe\_Ratownictwo\_Medyczne
\bibitem{pa} 
Encyklopedia Internetowa - Wikipedia, hasło: "112 (numer alarmowy)",
https://pl.wikipedia.org/wiki/112\_(numer\_alarmowy)




\end{thebibliography}

%----------------------------------------------------------------------------------------

\end{document}